% Innledning

\chapter{Innledning} % Chapter title

\label{ch:innledning} % For referencing the chapter elsewhere, use \autoref{ch:introduction} 
%----------------------------------------------------------------------------------------

Denne prosjektrapporten er skrevet i forbindelse med gruppe 4 sitt arbeid i emnet TTK4851 Eksperter i Team, landsby for Robotikk og Menneske. Arbeidet foregikk våren 2015, fra oppstart 7. januar, til levering 29. april. I dette kapittelet fokuseres det på bakgrunn, motivasjon, begrensninger, bidrag og videre disposisjon av oppgaven. 

\section{Bakgrunn for oppgaven}

%\begin{itemize}
%\item Hva er behovet?
%\item Hvordan har oppgaven vært løst tidligere?
%\item Hvordan går vårt prosjekt inn i "den store sammenhengen"?
%\item Hva bygger dette prosjektet på (for eksempel en masteroppgave, et konkret annet arbeid)
%\end{itemize}

\subsection{Behov}

En autonom robot er en plattform med mange bruksområder. Denne rapporten ser på mulighetene med å erstatte den tradisjonelle handlevognen med en robot. På denne måten slipper kunden å dytte rundt på vognen sin selv. Til korte handleturer med lette eller små varer vil det ikke være noe særlig behov for en slik robot, da en kurv eller vogn kan gjøre jobben billigere og bedre. Hvis man derimot skal handle større eller tyngre varer, f.eks. på IKEA eller store varehus, kan en robot gjøre handleprosessen mye enklere. Det er også særlig i tilfeller hvor personen som handler er handikappet, gammel eller på annen måte er i dårlig stand til å trille eller bære varer selv, en butikkrobot vil være til mye hjelp. 

\subsection{Tidligere arbeid}
Innenfor verden av roboter og droner har det i de siste årene blitt en økende trend i roboter som beveger seg rundt av seg selv. I artikkelen til ukebladet Wired \citep{next_big_trend} kan en lese om roboter som følger mennesker ved hjelp av ultralydsensorer eller lys-sensorer. For å unngå hindringer er det både eksperimentert med lasersensorer og ultrasoniske løsninger, samt tredimensjonal kartlegging ved bruk av laser. Robotene, som artikkelen omtaler, er planlagt å kunne brukes kommersielt. Eksempel på oppgaver er lagerarbeid ved varehus og fabrikker, caddy på en golfbane og ved jordbruks- og industrimiljøer.

\subsection{Den store sammenhengen}
En rekke aktører har allerede utviklet roboter som følger etter deg og bærer ting for deg. Disse robotene er vanligvis dyre, og som regel forbeholdt sektorer som industri og helse. Vi ønsker derimot å presentere et forslag til forbrukerne, nemlig en robot som skal kunne brukes av kunden i en butikk eller et varehus. Det er derfor viktig at løsningen passer i slike lokaler, og at den heller ikke har for store kostnader. Prisen på produktet er svært viktig, ettersom butikker i dag ikke bruker mange kroner på en vanlig handlevogn. %Det er også viktig å presisere at den løsningen som blir presentert i denne rapporten kun er en prototype, og ikke et ferdig produkt.

\section{Motivasjon}
Det har alltid vært et mål for gruppen å kunne bruke de respektive fagfeltene til hvert enkelt gruppemedlem, for deretter å bruke denne kompetansen til å få et best mulig produkt. Halvparten av gruppemedlemmene var personer med utdanning innen teknisk kybernetikk, mens resten av medlemmene satt med teknisk kunnskap fra andre retninger. Dette gjorde at vi hadde gode forutsetninger til å kunne gjøre noe praktisk i løpet av arbeidsperioden. Gruppesammensetningen er derfor en viktig årsak til at vi som gruppe havnet på ideen om å lage en menneskefølgende butikk-robot. I tillegg til kompetansen som lå til grunn i gruppa, var det også en stor enighet om at det var en oppgave som kunne være morsom å gjennomføre.

\section{Begrensninger}
%\begin{itemize}
%\item Har vi fokusert på biter av oppgaven?
%\item Har det vært ressursbegrensninger som gjør at vi ikke har gjort ideelle valg?
%\end{itemize}

\subsection{Oppdeling av oppgave}
Vi har fra dag én hatt et stort fokus på å gjøre oppgaven så modulær som mulig. Stifinning, sensorer, kollisjon-deteksjon og robot-kontroll ble naturlige biter i et større puslespill. Fokuset på gjøre prosjektet modulært kom først og fremst fra ønsket om å kunne jobbe parallelt med ulike deler. Det var fra starten av veldig høyt fokus på at disse skulle kunne jobbes med selvstendig og uavhengig, men på grunn av begrensninger på utstyr ble gruppen tvunget til å smelte bitene sammen.

\subsection{Robot}
Til dette prosjektet, fikk gruppen tilgang på én utviklingsrobot av typen Pioneer 3-DX. Dette har vært problematisk for gruppen, da en hele tiden har hatt et ønske om å kunne jobbe parallelt med både stifinning, sensorer, kollisjon-deteksjon og robot-kontroll. Roboten og sensorene lar bare én datamaskin være tilkoblet om gangen, så hvis noen ønsket å jobbe med forskjellige oppgaver, ble gruppen nødt til å vente på tur for å kunne teste løsningene. Gruppen har jobbet rundt begrensningene ved å lese og lære om teknologiene og metodene som kan brukes på forhånd av faktisk utvikling, men det hadde definitivt vært enklere om det hadde vært tilgang på mer utstyr.

\subsection{Kompetanse}
Til tross for at var flere på gruppen som hadde relevant kompetanse innenfor både robotikk og programvare, så var hadde det ingen som hadde erfaring med Pioneer 3-DX eller robot operativsystemet (ROS). Vi fikk også inntrykk av at det ikke var så mange på universitetet som hadde god erfaring med roboten. Alt dette medførte at gruppen brukte mye tid på å bli kjent med utstyret og gjennomføre opplæring av hverandre. 

\subsection{Tid}
Eksperter i team er et prosjekt med begrenset tid, og det er derfor vanskelig å kunne ferdigstille et avansert teknisk produkt på den tiden som er avsatt. I tillegg til dårlig tid, er det også kun avsatt onsdager for å jobbe med emnet. Mye blir glemt på en uke, noe som gjør at en må bruke en del tid hver landsbydag på å komme inn i det man gjorde forrige gang. Denne oppgaven hadde uten tvil vært enklere hvis landsbydagene hadde vært sammenhengene. 

% Litt usikker på hvor dette kommer fra ---> %Før den endelige planen for gjennomføringen av prosjektet ble bestemt, var det flere mulige andre metoder som ble diskutert for å hente posisjonsdata. I utgangspunktet hadde gruppen valgt å finne posisjon ved hjelp av Bluetooth beacons ved å triangulere signalet. Bruk av signalstyrke for å finne avstand var en mulighet, men for unøyaktig for denne typen bruk. En annen metode er å bruke fasen til signalet for å avgjøre avstanden og dermed posisjon. Denne metoden var utilgjengelig for gruppen og kunne ikke anvendes. 

\section{Bidrag}
%Kort oppsummering av det nye / smarte som er gjort.
Vårt bidrag er en prototype og ikke et ferdigstilt produkt. Prototypen er en autonom robot som bruker et avansert infrarødt kamerasystem for å kunne plassere et menneske og en robot i rommet. Koordinater til person og robot blir behandlet slik at roboten kan følge etter mennesket. Roboten har også en laser-scanner som kan kartlegge omgivelsene og detektere objekter rundt seg. Hvis objekter er et hinder for roboten vil den finne veien til målet uten å skade seg selv eller objektet. 

\section{Disposisjon av oppgaven}
\todo[caption=Disposisjon]{Disposisjon av oppgaven: Se over at dette stemmer overens før levering}
% Kort (1 – 3 setninger) om hva som er i hvert kapittel.

% \paragraph{Litteraturstudie} Formålet med dette kapittelet av rapporten er å kunne vise til kunnskap og informasjonskilder gruppen har brukt igjennom prosjektiden. Dette vil være en oppsummering av alle litteraturkilder og forklare hva vi tok med oss videre.

\paragraph{Teori} Formålet med dette kapittelet er å gjøre rede for tidligere arbeid innenfor fagfeltet og samle bakgrunnsinformasjon som er nødvendig for å forstå det arbeidet som utføres i oppgaven. Kapittelet inneholder fagrelatert informasjon og tekniske begreper som brukes i løsning og implementasjon.

\paragraph{Løsning} Hensikten med dette kapittelet er å presentere ideen, krav\-ene og målene gruppen satte seg. 

\paragraph{Implementasjon} Formålet med dette kapittelet er å presentere vår implementasjon av løsningene av kravene vi skulle oppnå. Dette kapittelet vil også inneholde begrunnelser for valg.

\paragraph{Resultater} Formålet med dette kapittelet er å ta for oss målene og resultater som viser til at vi har oppnådd kravene til oppgaven.

\paragraph{Diskusjon} Formålet med dette kapittelet er å diskutere resultat\-ene, først og fremst mot teori og implementasjon.

\paragraph{Konklusjon} I dette kapittelet ønsker vi å uttrykke konklusjonene for vårt system.

\paragraph{Videre arbeid} I dette kapittelet vil vi legge frem mulig videre arbeid og hva som eventuelt må gjøres før dette kan iverksettes.