% Resultater

\chapter{Resultater} % Chapter title

\label{ch:resultater} % For referencing the chapter elsewhere, use \autoref{ch:mathtest}

% Formålet med dette kapittelet er å ta for oss målene og resultater som viser til at vi har oppnådd kravene til oppgaven.
%----------------------------------------------------------------------------------------
Kravene til systemet ble innført som moduler, og disse modulene ble implementert hver for seg. Disse omfatter styring og følging, kollisjonsdeteksjon og sikkerhet. Posisjonering av robot og menneske ble realisert ved bruk av infrarøde kamera og programvare fra OptiTrack. Stifinning, laser-mapping, sonar og generell motorkontroll av robot ble implementert ved bruk av programvare-rammeverket ROS. 
Vår prototype har liten form for sikkerhetstiltak utenom kollisjonsdetekesjon, hvor det var ønsket å kunne se på muligheter for å håndtere defekte sensorer og mulige situasjoner som ville ødelegge for handlevognen under kjøring.

De fleste modulene ble testet sammen, men det ble derimot ikke tid til å lage et fullstendig system som oppfylte spesifikasjonen vi hadde for konseptet autonom handlevogn.

En demonstrasjonsvideo\footnote{\url{https://www.youtube.com/watch?v=3kurIgqk0us}} ble laget for illustrere hvordan produktet skulle fungere i praksis. I videoen blir roboten styrt ved hjelp av en xbox-kontroller.  

\todo{Hva oppnådde vi med prosjektet?}