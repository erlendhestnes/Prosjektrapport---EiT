% Videre arbeid

\chapter{Videre arbeid} % Chapter title

\label{ch:viderearbeid} % For referencing the chapter elsewhere, use \autoref{ch:mathtest}

%  I dette kapittelet vil vi legge frem mulig videre arbeid og hva som eventuelt må gjøres før dette kan iverksettes. 
%----------------------------------------------------------------------------------------

Gjennom dette prosjektet har gruppen presentert et mulig konsept for en autonom handlevogn robot. Mye må likevel gjøres hvis produktet skal videreføres og eventuelt kommersialiseres. Dette kapittelet presenterer ulike forslag og synspunkter på en eventuell videreføring av konseptet.

Som tidligere nevnt i denne rapporten, så er pris en viktig suksessfaktor for et slikt produkt. Dagligvarebutikkene er på mange måter konservative, og lite villig til å investere dyrt i nye løsninger. Det er dessuten lite trolig at en utskiftning av ordinære handlevogner til fordel for autonome vil gi en stor økonomisk avkastning. Produktet bør først og fremst ansees som en mulighet for butikkene til å innføre et nytt sortiment av varer. En autonom handlevogn vil nemlig kunne åpne for å handle varer som tidligere ikke hadde vært mulig, på grunn av vekt eller størrelse. Dette er viktige betraktninger som bør tas hensyn til i en videre utvikling.

Posisjonering ble i dette prosjektet gjort ved hjelp av et motion capture system. Et slikt system er veldig presist, men er dyrt å anskaffe, spesielt for butikker som har et stort areal. En alternativ løsning til et slikt system anbefales derfor på sterkeste. Radiobasert posisjonering ved faseforskyvning kan f.eks. være et godt alternativ. Det ville også vært naturlig å forbedre posisjoneringen ved bruk av SLAM.

Et annet aspekt en må tenke på, er robotens intelligens. Bør den kunne følge dumt etter brukeren, eller bør den ha mulighet til å ta enkelte beslutninger på egen hånd. I et butikklokale kan det fort oppstå hindringer. Disse kan være lett for en person å unngå, men verre for en stor og tung robot. Roboten bør kanskje derfor ha en mulighet til å kunne foreta smarte valg, ved å finne alternative ruter og styre unna hindringer.

Det er for tiden et høyt fokus og satsning på selvkjørende biler. Mye av teknologien som utvikles på det området, vil kanskje være naturlig å ta i bruk i et slikt prosjekt.